
\documentclass[11pt]{article}
\renewcommand{\baselinestretch}{1.05}
\usepackage{amsmath}
\topmargin0.0cm
\headheight0.0cm
\headsep0.0cm
\oddsidemargin0.0cm
\textheight23.0cm
\textwidth16.5cm
\footskip1.0cm

\begin{document}

% Title
\title{Solving Differential Equations}
\author{Krishn Ramesh}
\maketitle



% Section 1
\section{Power Series Solution of $y''+p(t)y'+q(t)y=0$}

This is a generalized ODE where $y=y(t)$. The solution is a power series function.

\subsubsection*{Ordinary Points}
A function $f(t)$ is called analytic at $t_0$ if it has a Taylor Series expansion i.e. all its derivatives from 1 to $\infty$ exist. For our DE $y''+p(t)y' +q(t) = 0$, $t_0$ is an ordinary point if both $p(t)$ and $q(t)$ are analytic. If this is the case, the solution below solves the DE.

\subsubsection*{Ordinary Point Solution}
\begin{enumerate}

\item Use the \textbf{trial function}:
$$
y(t)= \sum \limits_{n=0}^{\infty} a_n(t-t_0)^n
$$

$t_0$ is an arbitrary center point. If not given, it is assumed to be 0. Our goal is to find $a_n$.

\item The \textbf{index shifted} derivatives of $y(t)$ are:

\begin{gather*}
y'(t)= \sum \limits_{n=1}^{\infty} a_n n (t-t_0)^{n-1} = \sum \limits_{n=0}^{\infty} a_{n+1} (n+1) (t-t_0)^n \\
y''(t)= \sum \limits_{n=2}^{\infty} a_n n (n-1)(t-t_0)^{n-2} = \sum \limits_{n=0}^{\infty} a_{n+2} (n+1) (n+2) (t-t_0)^n
\end{gather*}

\item Substitute back into the DE.
\item Collect the $(t-t_0)^n$ terms to obtain a \textbf{recurrence relation}. It should look like:

$$
a_{n+2} = (...)a_n
$$

\item If possible, generalize the recurrence relation into even and odd cases. Write out $a_0$ through $a_3$ and simplify to look for patterns. The generalized even case will be $a_{2n}$ and the odd case will be $a_{2n+1}$.

\item The final solution is:

$$
y(t) = \sum \limits_{n=0}^{\infty} a_{2n} (t-t_0)^{2n} + \sum \limits_{n=0}^{\infty} a_{2n+1} (t-t_0)^{2n+1}
$$

\end{enumerate}

\subsubsection*{Singular Points}
If $t_0$ is not ordinary, it is singular.
 \paragraph{Regular Singular Point} if $(t-t_0)p(t)$ and $(t-t_0)^2 q(t)$ are both analytic. Alternatively, you can check if $\displaystyle{ \lim_{t\to t_0} (t-t_0)p(t) }$ and $\displaystyle{ \lim_{t\to t_0} (t-t_0)^2 q(t) }$ exist.
 \paragraph{Irregular Singular Point} otherwise.

\subsubsection*{Regular Singular Point Solution aka Method of Frobenius}
Pretty much the same as the Ordinary Point Solution except you use a slightly different trial function.

\begin{enumerate}

\item Use the trial function:
$$y(t)= \sum \limits_{n=0}^{\infty} a_n(t-t_0)^{n+\lambda}$$

The goal is to find $\lambda$ and $a_n$.

\item The derivatives of $y(t)$ are:

\begin{gather*}
y'(t)= \sum \limits_{n=1}^{\infty} a_n (n+\lambda) (t-t_0)^{n+\lambda-1} \\
y''(t)= \sum \limits_{n=2}^{\infty} a_n (n+\lambda) (n+\lambda-1)(t-t_0)^{n+\lambda-2}
\end{gather*}

\item Index shift the derivatives so that the exponents are the same.
\item You will find some of your series start at $n=-1$ while others start at $n=0$. Unroll the $n=-1$ term so that all the series start at 0 and you can combine them.
\item Extract a quadratic equation for $\lambda$. This comes from the unrolling of the $n=-1$ term from the previous step.
\item Solve for $\lambda_1$ and $\lambda_2$.
\item For each $\lambda$, find the recurrence relation.
\item Write out $a_1$, $a_2$, $a_3$ for each lambda.
\item The linearly independent solution for each $\lambda$ is:

$$y_{1,2}(t) = (t-t_0)^{\lambda_{1,2}}[a_0 + a_1t + a_2t^2 + ... ] $$

\item The general solution is $y(t) = c_1y_1(t) + c_2y_2(t)$

\end{enumerate}

\paragraph{Note} If $\lambda_1 - \lambda_2$ is an integer, you will only get 1 linearly independent solution. The other one is found using the Extended Method of Frobenius.


% Section 2
\section{Cauchy-Euler Equation $t^2y''+aty'+by=0$}
This is a special case of the DE in section 1.

\subsubsection*{Solution}
\begin{enumerate}

\item Use the trial function:
$$ y(t) = |t|^{\lambda} $$

\item Find its derivatives:
\begin{gather*}
y'(t)= \lambda t^{\lambda -1} \\
y''(t)= \lambda (\lambda-1) t^{\lambda-2}
\end{gather*}

\item Substitute into the DE.
\item Collect terms by exponent.
\item Extract quadratic equation for $\lambda$ (indicial equation) and solve for $\lambda_1$ and $\lambda_2$.
\item Only 2 cases we care about:
\paragraph{Distinct real roots} $y_1(t) = t^{\lambda_1}$ and $y_2(t) = t^{\lambda_2}$
\paragraph{Repeated real roots} $y_1(t) = t^{\lambda_1}$ and $y_2(t) = t^{\lambda_2}\ln(t)$

\item The general solution in both cases is $y(t) = c_1y_1(t)+c_2y_2(t)$
\end{enumerate}



% Section 3
\section{Bessel's DE $t^2y''+ty'+(t^2-\nu^2)y=0$}
This DE is usually arrived at from other DEs through change of variables, which is most of the work. Actually solving the DE is just a matter of stating the general solution.

\subsubsection*{Solution}
You only get one linearly independent solution:
$$ y(t) = c_1J_0(t) + c_2Y_0(t) $$

$J_0(t)$ is called the Bessel's function of the 1st kind of order $\nu=0$.
$Y_0(t)$ is called the Bessel's function of the 2nd kind of order 0.


% Section 4
\section{Heat Equation $u_t=\kappa u_{xx}$}

\subsection{Homogenous Boundary Conditions}
This means that all the BCs equal 0. There are two types of BCs.

\paragraph{Frozen ends} $u(0,t) = u(l,t) = 0$
\paragraph{Insulated ends} $u_x(0,t) = u_x(l,t) = 0$

\subsubsection*{Solution}

\begin{enumerate}

\item Separation of Variables: assume the solution $u(x,t) = X(x)T(t)$
\item Substitute into the PDE to obtain:

$$ \frac{T'(t)}{\kappa T(t)} = \frac{X''(x)}{X(x)} = \sigma $$

\item The solution $T(t)$ is:
$$ T(t) = Ce^{\kappa \sigma t}$$

\item Assuming $\sigma < 0$, the solution $X(x)$ is:
$$ X(x) = A\cos(\sqrt{-\sigma}x) + B\sin(\sqrt{-\sigma}x) $$

\item Plug the BCs into $X(x)$ to find $\sigma$:
$$ \sigma = -(\frac{n\pi}{l})^2 $$
\item Substitute $X_n(x)$ and $T_n(t)$ back into $u(x,t)$ to obtain:

$$ u(x,t) = \sum \limits_{n=0}^{\infty} u_n(x,t) = \sum \limits_{n=0}^{\infty} B_n \sin(\frac{n\pi}{l}x)e^{-(\frac{n\pi}{l})^2\kappa t}$$
For frozen ends, you end up with just $\sin$. For insulated ends, you end up with just $\cos$.

\item To find $B_n$, sub in the IC $u(x,0)=f(x)$ to find that it resembles the Fourier series, and so:
$$ B_n = \frac{1}{l} \int_{-l}^{l} f(x) \sin(\frac{n\pi}{l}x) dx$$

\item If the integral is straightforward, solve it, otherwise get it into one of the following forms through trig identities:

\begin{align*}
	\frac{1}{l} \int_{-l}^{l} cos(\frac{n\pi}{l}x)cos(\frac{m\pi}{l}x) &=
		\begin{cases}
			1, & n=m \\
			0, & n\neq m
		\end{cases}\\
\frac{1}{l} \int_{-l}^{l} cos(\frac{n\pi}{l}x)sin(\frac{m\pi}{l}x) &= 0 \\
\frac{1}{l} \int_{-l}^{l} sin(\frac{n\pi}{l}x)sin(\frac{m\pi}{l}x) &=
		\begin{cases}
			1, & n=m \\
			0, & n\neq m
		\end{cases}\\
\end{align*}

\item Plug in $B_n$ back into $u(x,t)$ to complete the solution.

\end{enumerate}

\subsection{Non-Homogenous Boundary Conditions}

Separation of variables cannot be used because not all the BCs are 0. The plan is to change variables until all the BCs are 0 and then the solution is the same as the homogenous case.

\subsubsection*{Solution}

\begin{enumerate}

\item Find the steady-state temperature, i.e. the temperature as $t \to \infty$:

\begin{align*}
&\implies u_t = 0 \\
&\implies u_{xx} = 0 \text{  (from PDE)} \\
&\implies u''_{ss}(x) = 0 \\
&\implies u'_{ss}(x) = a \\
&\implies u_{ss}(x) = ax+b
\end{align*}

\item Sub in the BCs to find $u_{ss}$.

\item Define $w(x,t)$ to be the transient temperature and:
$$ u(x,t) = w(x,t) + u_{ss}(x) $$

\item Rewrite the I/BVP in terms of $w$. The PDE remains unchanged but the BCs should now be 0.
\item Solve the new I/BVP for $w(x,t)$ using the same method as the previous section on homogenous BCs.
\item Write the final solution in terms of $u(x,t)$.

\end{enumerate}

\subsection{Non-Homogenous Heat Problem $u_t = \kappa u_{xx} + g(x,t)$}

\subsubsection*{Solution}

\begin{enumerate}

\item Assume g can be expressed as a Fourier sine series (if frozen ends) or cosine series (if insulated ends).

\begin{align*}
g(x,t) = \sum \limits_{n=0}^{\infty} g_n(t) \cos(\frac{n\pi}{l}x) \\
g_n(t) = \frac{1}{l} \int_{-l}^{l} g(x,t) \cos(\frac{n\pi}{l}x) dx
\end{align*}

\item The solution is of the form:

$$ u(x,t) = \sum \limits_{n=0}^{\infty} T_n(t) \cos(\frac{n\pi}{l}x) $$

The goal is to find $T_n(t)$.

\item Plug in the $u(x,0)$ BC to find $T_n(0)$, which serves as initial conditions.

\item Sub everything into the PDE to end up with:

$$ T_n'(t) + \kappa (\frac{n\pi}{l})^2 T_n(t) = g_n(t) $$

\item Solve the n DEs from the previous equation to solve the problem.

\end{enumerate}


% Section 5
\section{Wave Problem $u_{tt}=c^2u_{xx}$}
Here, $0<x<l, t>0$.
\paragraph{BCs} $u(0,t) = u(l,t) = 0$
\paragraph{ICs} $u(x,0) = f(x)$ and $u_t(x,0) = g(x)$

\subsubsection*{Solution}

\begin{enumerate}

\item Separation of Variables: assume the solution $u(x,t) = X(x)T(t)$
\item Substitute into the PDE to obtain:

$$ \frac{T''(t)}{c^2T(t)} = \frac{X''(x)}{X(x)} = \sigma $$

\item The solution $X(x)$ is:
$$ X(x) = c_1\cos(\sqrt{-\sigma}x) + c_2\sin(\sqrt{-\sigma}x) $$

\item Plug the BCs into $X(x)$ to find $\sigma$:
$$ \sigma = -(\frac{n\pi}{l})^2 $$

\item The solution $T(t)$ is:
$$ T_n(t) = a_n\cos(\frac{n\pi}{l}ct) + b_n\sin(\frac{n\pi}{l}ct) $$

\item Substitute $X_n(x)$ and $T_n(t)$ back into $u(x,t)$ to obtain:

$$ u(x,t) = \sum \limits_{n=0}^{\infty} \left( a_n\cos(\frac{n\pi}{l}ct) + b_n\sin(\frac{n\pi}{l}ct) \right) \sin(\frac{n\pi}{l}x)$$

\item Plug in the ICs to get $a_n, b_n$.

\begin{gather*}
a_n = \frac{1}{l} \int_{-l}^{l} f(x) \cos(\frac{n\pi}{l}x) dx \\
b_n = \frac{1}{n\pi c} \int_{-l}^{l} g(x) \sin(\frac{n\pi}{l}x) dx
\end{gather*}
Take the odd extension of $g(x)$ and integrate:
$$
b_n = \frac{2}{n\pi c} \int_{0}^{l} g(x) \sin(\frac{n\pi}{l}x) dx
$$
\end{enumerate}



% Section 6
\section{Traveling Wave $u_{tt}=c^2u_{xx}$}
The difference between the Wave Problem and the Traveling Wave is that here, $-\infty<x<\infty$. There are no boundary conditions.
\paragraph{ICs} $u(x,0) = f(x)$ and $u_t(x,0) = g(x)$

\subsubsection*{Solution}
$$u(x,t) = \frac{1}{2} [ f(x-ct) + f(x+ct)] + \frac{1}{2c} \int_{x-ct}^{x+ct} g(s) ds $$

% Section 7
\section{Dirichlet Problem $u_{xx}+u_{yy}+y_{zz}=0$}

\subsection{All corners 0}
\subsubsection*{Solution}
\begin{enumerate}

\item Separation of Variables: assume the solution $u(x,y) = X(x)Y(y)$
\item Substitute into the PDE to obtain:

$$ -\frac{X''(x)}{X(x)} = \frac{Y''(y)}{Y(y)} = \sigma $$

\item Assuming $\sigma > 0$,  the solution $X(x)$ is:
$$ X(x) = a\cos(\sqrt{\sigma}x) + b\sin(\sqrt{\sigma}x) $$

\item Plug in the corners into $X(x)$ to find $\sigma$:
$$ \sigma = (\frac{n\pi}{l})^2 $$

\item The solution $Y(y)$ is:
$$ Y_n(y) = c_1\cosh(\frac{n\pi}{l}y) + c_2\sinh(\frac{n\pi}{l}y) $$

\item Plug in the corners to find the constants.

\item Substitute $X_n(x)$ and $Y_n(y)$ back into $u(x,y)$ to obtain:

$$ u(x,t) = \sum \limits_{n=0}^{\infty} b_n\sin(\frac{n\pi}{l}x) \sinh(\frac{n\pi}{l}y) $$

\item Plug in the BCs to get $b_n$.
\end{enumerate}

\subsection{Corners non-zero}
\subsubsection*{Solution}
\begin{enumerate}

\item Make the corners 0 by using the change of variables:

$$ U(x,y) = u(x,y) - \nu(x,y) $$

where

$$ \nu(x,y) = \alpha_1 +\alpha_2x + \alpha_3y + \alpha_4xy $$

\item Setting $u=\nu$ at the corners results in a 4x4 system.

\item Once $\nu$ is known, write the new problem in terms of $U(x,y)$ and solve using the same method as above.

\end{enumerate}

\end{document}
